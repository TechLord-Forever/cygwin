%\documentclass[a4paper]{article}
\documentclass{article}
\usepackage{graphicx}
\usepackage{subfigure}
\usepackage{fancyvrb}
%\usepackage{times}
\usepackage[latin1]{inputenc}
\usepackage{url}

%\usepackage{lineno}
%\linenumbers
%\renewcommand{\baselinestretch}{1.5}

% Change url font to textsf (and check what breaks in PDF/HTML/...)

\fvset{xleftmargin=3em,commandchars=\\\{\}}

\newcommand{\quilt}[1]{\textsf{quilt #1}}
\newcommand{\sh}[1]{\textsl{#1}}
\newcommand{\prog}[1]{\textit{#1}}

\title{How To Survive With Many Patches\\
{\Large or}\\
Introduction to Quilt\footnote{
	Quilt is a GPL licensed project hosted on GNU Savannah. Some ideas
	for this document were taken from \textit{docco.txt} in
	Andrew Morton's patch management scripts package~\cite{akpm02}.
	The text in the examples was taken from \textit{A Midsummer
	Night's Dream} by William Shakespeare.
}}
\author{Andreas Gr�nbacher, SuSE Labs \\
%\em{SUSE Labs, SUSE LINUX AG} \\
{\normalsize agruen@suse.de}
}
%\date{}

\begin{document}

\maketitle

\thispagestyle{empty}

\begin{abstract}
After looking at different strategies for dealing with software packages
that consist of a base software package on top of which a number of
patches are applied, this document introduces the script collection
\textit{quilt,} which was specifically written to help deal with
multiple patches and common patch management tasks.
\end{abstract}

\section{Introduction}

% Prerequisites: UNIX, patches, using GNU diff and GNU patch.
% Why patches in the first place?

In the old days, vendor specific software packages in the open source
world consisted of a file with the official version of the software,
plus a patch file with the additional changes needed to adapt the
package to specific needs. The official software package was usually
contained in a \textsf{package.tar.gz} file, while the patch was found
in \textsf{package.diff.} Instead of modifying the official
package sources, local changes were kept separate. When building the
software package, the tar archive was extracted, and the patch was
applied.

Over time, the patch file ended up containing several independent
changes. Of those changes, some were integrated into later versions of
the software, while other add-ons or adaptations remain external. Whenever
a new official version was integrated, the patch needed to be revised:
changes that were already integrated in the official version needed to
be split from changes that were not.

A big improvement was to allow multiple patches in a vendor package,
and this is also how patches are handled today: a number of
patches is applied on top of each other. Each patch usually consists of
a logically related set of changes. When some patches get integrated
upstream, those patches can simply be removed from the vendor specific
package. The remaining patches frequently continue to apply cleanly.
Some of the remaining patches may have to be maintained across a range
of upstream versions because they are too specific for the upstream
software package, etc. These patches often get out of sync, and need to
be updated.

For the majority of packages, the number of patches remains relatively
low, so maintaining those patches without tools is feasible. A number of
packages have dozens of patches, however. At the extreme end is the
kernel source package (kernel-source-\textit{2.4.x}) with more than
1\,000 patches.  The difficulty of managing such a vast number of
patches without tools can easily be imagined.

This document discusses different strategies of dealing with large sets
of patches. Patches are usually generated by the \prog{diff} utility,
and applied with the \prog{patch} utility.  Different patch file formats are
defined as part of the specification of the \prog{diff} utility in
POSIX.1~\cite{posix-2001-diff}.  The most commonly used format today,
\textit{unified diff,} is not covered by POSIX.1, however.  A good
description of patch file formats is found in the \prog{GNU diff} info
pages~\cite{info-diff}.

The question we try to answer in this document is how patches are best kept
up to date in face of changes both to the upstream software package, and
to the patches that precede them.  After looking at some existing
approaches, a collection of patch management scripts known as
\textit{quilt} is described~\cite{quilt}, starting with basic concepts,
and progressing towards more advanced tasks.

% - quilt
% (wet people's mouths about the features)

% How exactly does this relate to many patches?

\section{Existing Approaches}
\label{sec:existing}

The minimal solution for updating a patch is to apply all preceding
patches.
%\footnote{ In the kernel CVS, we have a a script called
%\textit{sequence-patch} that simply applies all patches up to a
%specified patch.  }
Then, a copy of the resulting source tree is created.\footnote{
	The two copies can also be hard-linked with each other, which
	significantly speeds up both the copying and the final
	``diffing''. If hard links are used, care must be taken that the
	tools used to update one copy of the source tree will create new
	files, and will not overwrite shared files. Editors like
	\prog{emacs} and \prog{vi}, and utilities like \prog{patch},
	support this.
} The next patch in the sequence of patches (which is the one to be
updated) is applied to only one of these source trees. This source tree
is then modified until it reflects the desired result. The new version of
the patch is distilled by comparing the two source trees with
\prog{diff}, and writing the result into a file.

This simple approach is rather error prone, and leaves much to be
desired. Several people have independently written scripts that
automate and improve upon this process.

A version control system like \prog{CVS} or \prog{RCS} may be a
reasonable alternative in some cases. The version control system is
brought in the state of the working tree with a number of patches
applied. Then the next patch is applied. After the working tree is
updated as required, a diff between the repository copy and the working
tree is created (with \prog{cvs diff}, etc). In this scenario the
version control system is used to store and compare against the old
repository version only. The full version control overhead is paid,
while only a small fraction of its functionality is needed. Switching
between different patches is not simplified.

% TODO: Mention some approaches here; RCS and CVS ...

One of the most advanced approaches is Andrew Morton's patch management
scripts~\cite{akpm02}.  The author of this document found that none of
the available solutions would scale up to the specific requirements of
the SUSE kernel-source package, and started to improve Andrew Morton's
scripts until they became what they are now~\cite{quilt}.

% - Re and Rd scripts (Czech scripts using RCS, replaces the
%   now-obsolete rpmpatch that supports one .dif only).
% - Werner's scripts

% What couldn't be done:
% - Patches in sub-directories
% - Many patches
% - Retaining documentation (akpm's scripts do part of this)

% Actually merging rejects is not handled; use tools like:
% - wiggle
% - Other merge tools (e.g., graphical ones)

\section{Quilt: Basic Concepts and Operation}
\label{sec:basic}

The remainder of this document discusses the script collection
\textit{quilt.}

With quilt, all work occurs within a single directory tree. Since
version 0.30, commands can be invoked from anywhere within the source
tree.  Commands are of the form ``\quilt{cmd},'' similar to CVS
commands. They can be abbreviated as long as the specified part of the
command is unique. All commands print some help text with ``\quilt{cmd
-h}.''

Quilt manages a stack of patches. Patches are applied incrementally on
top of the base tree plus all preceding patches. They can be pushed
on top of the stack (\quilt{push}), and popped off the stack
(\quilt{pop}).  Commands are available for querying the contents of the
series file (\quilt{series}, see below), the contents of the stack
(\quilt{applied}, \quilt{previous}, \quilt{top}), and the patches that
are not applied at a particular moment (\quilt{next}, \quilt{unapplied}).
By default, most commands apply to the topmost patch on the stack.

When files in the working directory are changed, those changes become
part of the working state of the topmost patch, provided that those
files are part of the patch. Files that are not part of a patch must be
added before modifying them so that quilt is aware of the original
versions of the files. The \quilt{refresh} command regenerates a patch.
After the refresh, the patch and the working state are the same.

Patch files are located in the \textsf{patches} sub-directory of the
source tree (see Figure~\ref{fig:dir-layout}). The \textsf{QUILT\_PATCHES}
environment variable can be used to override this location. The
\textsf{patches} directory may contain sub-directories.
\textsf{patches} may also be a symbolic link instead of a directory.

A file called \textsf{series} contains a list of patch file names that
defines the order in which patches are applied. Unless there are means
by which series files can be generated automatically (see
Section~\ref{sec:rpm}), they are usually provided along with a set of
patches. In \textsf{series}, each patch file name is on a separate line.
Patch files are identified by pathnames that are relative to the
\textsf{patches} directory; patches may be in sub-directories below the
\textsf{patches} directory.  Lines in the series file that start with a
hash character (\texttt{\#}) are ignored.  When quilt adds, removes, or
renames patches, it automatically updates the series file.  Users of
quilt can modify series files while some patches are applied, as long as
the applied patches remain in their original order.

Different series files can be used to assemble patches in different ways,
corresponding for example to different development branches.

\begin{figure}
\begin{center}
\begin{minipage}{6cm}
\begin{small}
\begin{Verbatim}
work/ -+- ...
       |- patches/ -+- series
       |            |- patch2.diff
       |            |- patch1.diff
       |            +- ...
       +- .pc/ -+- applied-patches
                |- patch1.diff/ -+- ...
                |- patch2.diff/ -+- ...
                +- ...
\end{Verbatim}
\end{small}
\end{minipage}
\caption{Quilt files in a source tree.}
\label{fig:dir-layout}
\end{center}
\end{figure}

Before a patch is applied (or ``pushed on the stack''), copies of all
files the patch modifies are saved to the \textsf{.pc/\textit{patch}}
directory.\footnote{
	The patch name with extensions stripped is used as the name of
	the sub-directory below the \textsf{.pc} directory.  \prog{GNU patch},
	which quilt uses internally to apply patches, creates backup
	files and applies the patch in one step.
} The patch is added to the list of
currently applied patches (\textsf{.pc/applied-patches}).  Later when a patch is regenerated
(\quilt{refresh}), the backup copies in \textsf{.pc/\textit{patch}} are
compared with the current versions of the files in the source tree
using \prog{GNU diff}.

Documentation related to a patch can be put at the beginning of a patch
file.  Quilt is careful to preserve all text that precedes the actual
patch when doing a refresh.

The series file is looked up in the root of the source tree, in the
patches directory, and in the \textsf{.pc} directory.  The first series
file that is found is used. This may also be a symbolic link, or a file
with multiple hard links.  Usually, only one series file is used for a
set of patches, so the \textsf{patches} sub-directory is a convenient
location.

While patches are applied to the source tree, the \textsf{.pc} directory
is essential for many operations, including taking patches off the stack
(\quilt{pop}), and refreshing patches (\quilt{refresh}). Files in the
\textsf{.pc} directory are automatically removed when they are no longer
needed, so usually there is no need to clean up manually.  The
\textsf{QUILT\_PC} environment variable can be used to override the
location of the \textsf{.pc} directory.

\section{An Example}

This section demonstrates how new patches are created and updated, and
how conflicts are resolved. Let's start with a short text file:

\begin{small}
\begin{Verbatim}
Yet mark'd I where the bolt of Cupid fell:
It fell upon a little western flower,
Before milk-white, now purple with love's wound,
And girls call it love-in-idleness.
\end{Verbatim}
\end{small}

New patches are created with \quilt{new}. A new patch automatically
becomes the topmost patch on the stack. Files must be added
with \quilt{add} before they are modified. Note that this is slightly
different from the CVS style of interaction: with CVS, files are in the
repository, and adding them before committing (but after modifying them)
is enough.  Files are usually added and immediately modified. The
command \quilt{edit} adds a file and loads it into the default editor.
(The environment variable \textsf{EDITOR} specifies which is the default
editor.  If \textsf{EDITOR} is not set, \prog{vi} is used.)

\begin{small}
\begin{Verbatim}
\sh{$ quilt new flower.diff}
Patch flower.diff is now on top
\sh{$ quilt edit Oberon.txt}
File Oberon.txt added to patch flower.diff
\end{Verbatim}
\end{small}

Let's assume that the following lines were added to \textsf{Oberon.txt}
during editing:

\begin{small}
\begin{Verbatim}
The juice of it on sleeping eye-lids laid
Will make a man or woman madly dote
Upon the next live creature that it sees.
\end{Verbatim}
\end{small}

The actual patch file is created (and later updated) with
\quilt{refresh}. The result is as follows:\footnote{
	Timestamps in patches are omitted from the output in the examples.
}

\begin{small}
\begin{Verbatim}
\sh{$ quilt refresh}
\sh{$ cat patches/flower.diff}
Index: example1/Oberon.txt
===================================================================
--- example1.orig/Oberon.txt
+++ example1/Oberon.txt
@@ -2,3 +2,6 @@
 It fell upon a little western flower,
 Before milk-white, now purple with love's wound,
 And girls call it love-in-idleness.
+The juice of it on sleeping eye-lids laid
+Will make a man or woman madly dote
+Upon the next live creature that it sees.
\end{Verbatim}
\end{small}

Now let's assume that a line in the text has been overlooked, and must
be inserted. The file \textsf{Oberon.txt} is already part of the patch
\textsf{flower.diff}, so it can immediately be modified in an editor.
Alternatively, \quilt{edit} can be used; it simply opens up the default
editor if the file is already part of the patch.

After the line is added, we use \quilt{diff -z} to see a diff of the
changes we made:

\begin{small}
\begin{Verbatim}
\sh{$ quilt diff -z}
Index: example1/Oberon.txt
===================================================================
--- example1.orig/Oberon.txt
+++ example1/Oberon.txt
@@ -2,6 +2,7 @@
 It fell upon a little western flower,
 Before milk-white, now purple with love's wound,
 And girls call it love-in-idleness.
+Fetch me that flower; the herb I shew'd thee once:
 The juice of it on sleeping eye-lids laid
 Will make a man or woman madly dote
 Upon the next live creature that it sees.
\end{Verbatim}
\end{small}

A diff of the topmost patch can be generated with \quilt{diff} without
arguments. This does not modify the actual patch file.  The changes are
only added to the patch file by updating it with \quilt{refresh}. Then
we remove the patch from the stack with \quilt{pop}:

\begin{small}
\begin{Verbatim}
\sh{$ quilt refresh}
Refreshed patch flower.diff
\sh{$ quilt pop}
Removing flower.diff
Restoring Oberon.txt

No patches applied
\end{Verbatim}
\end{small}

Next, let's assume that \textsf{Oberon.txt} was modified ``upstream'':
The word \textit{girl} did not fit very well, and so it was replaced
with \textit{maiden.} \textsf{Oberon.txt} now contains:

\begin{small}
\begin{Verbatim}
Yet mark'd I where the bolt of Cupid fell:
It fell upon a little western flower,
Before milk-white, now purple with love's wound,
And maidens call it love-in-idleness.
\end{Verbatim}
\end{small}

This causes \textsf{flower.diff} to no longer apply cleanly.  When we
try to push \textsf{flower.diff} on the stack with \quilt{push}, we get
the following result:

\begin{small}
\begin{Verbatim}
\sh{$ quilt push}
Applying flower.diff
patching file Oberon.txt
Hunk #1 FAILED at 2.
1 out of 1 hunk FAILED -- rejects in file Oberon.txt
Patch flower.diff does not apply (enforce with -f)
\end{Verbatim}
\end{small}

Quilt does not automatically apply patches that have rejects. Patches
that do not apply cleanly can be ``force-applied'' with \quilt{push -f},
which leaves reject files in the source tree for each file that has
conflicts. Those conflicts must be resolved manually, after which the
patch can be updated (\quilt{refresh}). Quilt remembers when a patch has
been force-applied. It refuses to push further patches on top of such
patches, and it does not remove them from the stack.  A force-applied
patch may be ``force-removed'' from the stack with \quilt{pop -f},
however.  Here is what happens when force-applying \textsf{flower.diff}:

\begin{small}
\begin{Verbatim}
\sh{$ quilt push -f}
Applying flower.diff
patching file Oberon.txt
Hunk #1 FAILED at 2.
1 out of 1 hunk FAILED -- saving rejects to file Oberon.txt.rej
Applied flower.diff (forced; needs refresh)
\end{Verbatim}
\end{small}

After re-adding the lines of verse from \textsf{flower.diff} to
\textsf{Oberon.txt}, we update the patch with \quilt{refresh}.

\begin{small}
\begin{Verbatim}
\sh{$ quilt edit Oberon.txt}
\sh{$ quilt refresh}
Refreshed patch flower.diff
\end{Verbatim}
\end{small}

Our final version of \textsf{Oberon.txt} contains:

\begin{small}
\begin{Verbatim}
Yet mark'd I where the bolt of Cupid fell:
It fell upon a little western flower,
Before milk-white, now purple with love's wound,
And maidens call it love-in-idleness.
Fetch me that flower; the herb I shew'd thee once:
The juice of it on sleeping eye-lids laid
Will make a man or woman madly dote
Upon the next live creature that it sees.
\end{Verbatim}
\end{small}

\section{Further Commands and Concepts}

This section introduces a few more basic commands, and then describes
additional concepts that may not be immediately obvious.  We do not
describe all of the features of quilt here since many quilt commands are
quite intuitive; furthermore, help text that describes the available
options for each command is available via \quilt{\textit{cmd} -h}.

The \quilt{top} command shows the name of the topmost patch.  The
\quilt{files} command shows which files a patch contains.  The
\quilt{patches} command shows which patches modify a specified file.
With our previous example, we get the following results:

\begin{small}
\begin{Verbatim}
\sh{$ quilt top}
flower.diff
\sh{$ quilt files}
Oberon.txt
\sh{$ quilt patches Oberon.txt}
flower.diff
\end{Verbatim}
\end{small}

The \quilt{push} and \quilt{pop} commands optionally take a number or
a patch name as argument. If a number is given, the specified number of
patches is added (\quilt{push}) or removed (\quilt{pop}). If a patch
name is given, patches are added (\quilt{push}) or removed (\quilt{pop})
until the specified patch is on top of the stack. With the \textsf{-a}
option, all patches in the series file are added (\quilt{push}), or all
applied patches are removed from the stack (\quilt{pop}).

\subsection{Patch Strip Levels}

Quilt assumes that patches are applied with a strip level of one (the
\textsf{-p1} option of \prog{GNU patch}) by default: the topmost directory
level of file names in patches is stripped off. Quilt remembers the
strip level of each patch in the \textsf{series} file. When generating a
diff (\quilt{diff}) or updating a patch (\quilt{refresh}), a different
strip level can be specified, and the series file will be updated
accordingly. Quilt can apply patches with an arbitrary strip level, and
produces patches with a strip level of zero or one. With a strip level
of one, the name of the directory that contains the working tree is used
as the additional path component. (So in our example,
\textsf{Oberon.txt} is contained in directory \textsf{example1}.) 

\subsection{Importing Patches}

The \quilt{import} command automates the importing of patches into the
quilt system. The command copies a patch into the \textsf{patches}
directory and adds it to the \textsf{series} file. For patch strip
levels other than one, the strip level is added after the patch file
name. (An entry for \textsf{a.diff} with strip level zero might read
``{\small \verb|a.diff -p0|}''.)

Another common operation is to incorporate a fix or similar that comes
as a patch into the topmost patch. This can be done by hand by first
adding all the files contained in the additional patch to the topmost
patch with \quilt{add},\footnote{
	The \prog{lsdiff} utility, which is part of the \textit{patchutils}
	package, generates a list of files affected by a patch.
} and then applying the patch to the working tree.  The \quilt{fold}
command combines these steps.

\subsection{Sharing patches with others}

For sharing a set of patches with someone else, the series file which
contains the list of patches and how they are applied, and the patches
themselves are all that's needed. The \textsl{.pc} directory only
contains quilt's working state, and should not be distributed. Make sure
that all the patches are up-to-date, and refresh patches when
necessary. The \textsf{--combine} option of \quilt{diff} can be used for
generating a single patch out of all the patches in the series file.

\subsection{Merging with upstream}

The concept of merging your patches with upstream is identical to applying
your patches on a more recent version of the software. 

Before merging, make sure to pop all your patches using \quilt{pop -a}.
Then, update your codebase. Finally, remove obsoleted patches
from the series file and \quilt{push} the remaining ones, resolve
conflicts and refresh patches as needed.

\subsection{Forking}
\label{sec:forking}

There are situations in which updating a patch in-place is not ideal:
the same patch may be used in more than one series file. It may also
serve as convenient documentation to retain old versions of patches, and
create new ones under different names. This can be done by hand by
creating a copy of a patch (which must not be applied), and updating the
patch name in the series file.

The \quilt{fork} command simplifies this: it creates a copy of the
topmost patch in the series, and updates the series file. Unless a patch
name is explicitly specified, \quilt{fork} will generate the following
sequence of patch names: \textsf{patch.diff}, \textsf{patch-2.diff},
\textsf{patch-3.diff},\dots

\subsection{Dependencies}
\label{sec:dependencies}

When the number of patches in a project grows large, it becomes
increasingly difficult to find the right place for adding a new patch in
the patch series. At a certain point, patches will get inserted at the
end of the patch series, because finding the right place has become too
complicated. In the long run, a mess accumulates.

To help avoid this by keeping the big picture, the \quilt{graph} command
generates \textit{dot} graphs showing the dependencies between
patches.\footnote{
	The \quilt{graph} command computes dependencies based on
	which patches modify which files, and optionally will also
	check for overlapping changes in the files. While the former
	approach will often result in false positives, the latter
	approach may result in false negatives (that is, \quilt{graph}
	may overlook dependencies).
} The ouput of this command can be visualized using the tools from AT\&T
Research's Graph Visualization Project (GraphViz,
\url{http://www.graphviz.org/}).  The \quilt{graph} command supports
different kinds of graphs.

\subsection{Advanced Diffing}

Quilt allows us to diff and refresh patches that are applied, but are not
on top of the stack (\quilt{diff -P \textit{patch}} and \quilt{refresh
\textit{patch}}). This is useful in several cases, for example, when
%\begin{itemize}
%
%\item When the topmost patch has been modified but the changes are not
%yet completed, refreshing the patch would leave a patch file that is in
%an inconsistent state. Without that, the patch cannot be removed from
%the stack, or else the changes would be lost.
%
%\item Popping patches and then pushing them again results in modified
%time stamps. This may trigger time consuming recompiles.
%
%\item It is simply convenient to be able to fix small bugs in patches
%further down in the stack without much ado.
%
%\end{itemize}
%
patches applied higher on the stack modify some of the files that this
patch modifies. We can picture this as a shadow which the patches higher
on the stack throw on the files they modify.  When refreshing a patch,
changes to files that are not shadowed (and thus were last modified by
the patch that is being refreshed) are taken into account. The
modifications that the patch contains for shadowed files will not be
updated.

The \quilt{diff} command allows us to merge multiple patches into one by
optionally specifying the range of patches to include (see \quilt{diff
-h}). The combined patch will only modify each file contained in these
patches once. The result of applying the combined patch is the same as
applying all the patches in the specified range in sequence.

Sometimes it is convenient to use a tool other than \prog{GNU diff} for
comparing files (for example, a graphical diff replacement like
\prog{tkdiff}). Quilt will not use tools other than \prog{GNU diff} when
updating patches (\quilt{refresh}), but \quilt{diff} can be passed the
\textsf{-{}-diff=\textit{utility}} argument. With this argument, the
specified utility is invoked for each file that is being modified with
the original file and new file as arguments. For new files, the first
argument will be \textsf{/dev/null}.  For removed files, the second
argument will be \textsf{/dev/null}.

When \quilt{diff} is passed a list of file names, the diff will be
limited to those files. With the \textsf{-R} parameter, the original and
new files are swapped, which results in a reverse diff.

Sometimes it is useful to create a diff between an arbitrary state of
the working tree and the current version. This can be used to create a
diff between different versions of a patch (see
Section~\ref{sec:forking}), etc. To this end, quilt allows us to take a
snapshot of the working directory (\quilt{snapshot}). Later, a diff
against this state of the working tree can be created with \quilt{diff
-{}-snapshot}.

Currently, only a single snapshot is supported. It is stored in the
\textsf{.pc/.snap} directory.  To recover the disk space the snapshot
occupies, it can be removed with \quilt{snapshot -d}, or by removing the
\textsf{.pc/.snap} directory manually.

\subsection{Working with RPM Packages}
\label{sec:rpm}

Several Linux distributions are based on the RPM Package
Manager~\cite{max-rpm}. RPM packages consist of a spec that defines how
packages are built, and a number of additional files like tar archives,
patches, etc.  Most RPM packages contain an official software package
plus a number of patches. Before these patches can be manipulated with
quilt, a series file must be created that lists the patches along with
their strip levels.

The \quilt{setup} command automates this for most RPM packages. When
given a spec file as its argument, it performs the \textsf{\%prep}
section of the spec file, which is supposed to extract the official
software package, and apply the patches. In this run, quilt remembers
the tar archives and the patches that are applied, and creates a series
file.  Based on that series file, \quilt{setup} extracts the archives,
and copies the patches into the \textsf{patches} sub-directory. Some
meta-information like the archive names are stored as comments in the
series file. \quilt{setup} also accepts a series file as argument (which
must contain some meta-information), and sets up the working tree from
the series file in this case.

\section{Customizing Quilt}

Upon startup, quilt evaluates the file \textsf{.quiltrc} in the user's
home directory, or the file specified with the \textsf{--quiltrc} option.
This file is a regular bash script. Default options can be passed to
any command by defining a \textsf{QUILT\_\textit{COMMAND}\_ARGS} variable
(for example, \textsf{QUILT\_DIFF\_ARGS="--color=auto"} causes the output
of \quilt{diff} to be syntax colored when writing to a terminal).

In addition to that, quilt recognizes the following variables:

\begin{description}

\item[\textsf{QUILT\_DIFF\_OPTS}]
Additional options that quilt shall pass to \prog{GNU diff} when
generating patches. A useful setting for C source code is
``\textsf{-p}'', which causes \prog{GNU diff} to show in the resulting
patch which function a change is in.

\item[\textsf{QUILT\_PATCH\_OPTS}]
Additional options that quilt shall pass to \prog{GNU patch} when
applying patches. (For example, some versions of \prog{GNU patch}
support the ``\textsf{--unified-reject-files}'' option for generating
reject files in unified diff style.

\item[\textsf{QUILT\_PATCHES}]
The location of patch files (see Section~\ref{sec:basic}).  This setting
defaults to ``\textsf{patches}''.

\end{description}

\section{Pitfalls and Known Problems}

As mentioned earlier, files must be added to patches before they can be
modified. If this step is overlooked, one of the following problems will
occur: If the file is included in a patch further below on the stack,
the changes will appear in that patch when it is refreshed, and for that
patch the \quilt{pop} command will fail before it is refreshed.  If the
file is not included in any applied patch, the original file in the
working tree is modified.

Patch files may modify the same file more than once. \prog{GNU patch}
has a bug that corrupts backup files in this case. A fix is available,
and will be integrated in a later version of \textit{GNU patch}.  The fix has
already been integrated into the SUSE version of \textit{GNU patch}.

There are some packages that assume that it's a good idea to remove all
empty files throughout a working tree, including the \textsf{.pc}
directory. The \textit{make clean} target in the linux kernel sources
is an example. Quilt uses zero-length files in \textsf{.pc} to mark
files added by patches, so such packages may corrupt the \textsf{.pc}
directory. A workaround is to create a symbolic link \textsf{.pc} in the
working tree that points to a directory outside.

It may happen that the files in the \textsf{patches} directory gets out of
sync with the working tree (e.g., they may accidentally get updated by
CVS or similar). Files in the \textsf{.pc} directory may also become
inconsistent, particularly if files are not added before modifying them
(\quilt{add} / \quilt{edit}). If this happens, it may be possible to
repair the source tree, but often the best solution is to start over
with a scratch working directory and the \textsf{patches} sub-directory.
There is no need to keep any files from the \textsf{.pc} directory in
this case.

% - Patches cannot automatically be reverse applied (?)
% - Does not auto-detect is a patch has been cleanly integrated
% - Speed

% - Push a patch that is not in the series file (after changing
%   applied-patches to include the full patch name)?

% - Other back-ends (RCS etc.)

% - Pop and push modify file timestamps, which causes recompiles.
%   Ccache ameliorates this.

% - patchutils: Very useful: lsdiff, filterdiff, etc.

\section{Summary}

We have shown how the script collection \textit{quilt} solves various
problems that occur when dealing with patches to software packages.
Quilt is an obvious improvement over directly using the underlying tools
(\prog{GNU patch}, \prog{GNU diff}, etc.), and offers many features not
available with competing solutions. Join the club!

The quilt project homepage is
\url{http://savannah.nongnu.org/projects/quilt/}. There is a development
mailing list at \url{http://mail.nongnu.org/mailman/listinfo/quilt-dev}.
Additional features that fit into quilt's mode of operation are always
welcome, of course.

\begin{thebibliography}{XX}

\bibitem{akpm02}
Andrew Morton: Patch Management Scripts,
\url{http://lwn.net/Articles/13518/} and
\url{http://www.zip.com.au/~akpm/linux/patches/patch-scripts-0.9}.

\bibitem{quilt}
Andreas Gr�nbacher et al.: Patchwork Quilt,
\url{http://savannah.nongnu.org/projects/quilt}.

\bibitem{posix-2001-diff}
IEEE Std. 1003.1-2001: Standard for Information Technology, Portable
Operating System Interface (POSIX), Shell and Utilities, diff
command, pp.~317. Online version available from the The Austin Common
Standards Revision Group, \url{http://www.opengroup.org/austin/}.

\bibitem{info-diff}
\textit{GNU diff} info pages (\textsf{info Diff}), section \textit{Output
Formats.}

\bibitem{max-rpm}
Edward C. Bailey: Maximum RPM: Taking the Red Hat Package Manager to the
Limit, \url{http://www.rpm.org/max-rpm/}.

\end{thebibliography}

% Add a "quick-reference card"?

\end{document}
