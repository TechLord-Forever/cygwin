Most documentation systems have special `see also' sections where links to other pieces of documentation can be inserted. Although doxygen also has a command to start such a section (See section \hyperlink{commands_cmdsa}{$\backslash$sa}), it does allow you to put these kind of links anywhere in the documentation. For $\mbox{\LaTeX}$ documentation a reference to the page number is written instead of a link. Furthermore, the index at the end of the document can be used to quickly find the documentation of a member, class, namespace or file. For man pages no reference information is generated.

The next sections show how to generate links to the various documented entities in a source file.\hypertarget{autolink_linkurl}{}\section{Links to web pages and mail addresses}\label{autolink_linkurl}
Doxygen will automatically replace any URLs and mail addresses found in the documentation by links (in HTML).\hypertarget{autolink_linkclass}{}\section{Links to classes.}\label{autolink_linkclass}
All words in the documentation that correspond to a documented class and contain at least one upper case character will automatically be replaced by a link to the page containing the documentation of the class. If you want to prevent that a word that corresponds to a documented class is replaced by a link you should put a \% in front of the word.\hypertarget{autolink_linkfile}{}\section{Links to files.}\label{autolink_linkfile}
All words that contain a dot ({\tt .}) that is not the last character in the word are considered to be file names. If the word is indeed the name of a documented input file, a link will automatically be created to the documentation of that file.\hypertarget{autolink_linkfunc}{}\section{Links to functions.}\label{autolink_linkfunc}
Links to functions are created if one of the following patterns is encountered: \begin{enumerate}
\item {\tt $<$functionName$>$\char`\"{}(\char`\"{}$<$argument-list$>$\char`\"{})\char`\"{}} \item {\tt $<$functionName$>$\char`\"{}()\char`\"{}} \item {\tt \char`\"{}::\char`\"{}$<$functionName$>$} \item {\tt ($<$className$>$\char`\"{}::\char`\"{})$^{\mbox{n}}$ $<$functionName$>$\char`\"{}(\char`\"{}$<$argument-list$>$\char`\"{})\char`\"{}} \item {\tt ($<$className$>$\char`\"{}::\char`\"{})$^{\mbox{n}}$ $<$functionName$>$\char`\"{}(\char`\"{}$<$argument-list$>$\char`\"{})\char`\"{}$<$modifiers$>$} \item {\tt ($<$className$>$\char`\"{}::\char`\"{})$^{\mbox{n}}$ $<$functionName$>$\char`\"{}()\char`\"{}} \item {\tt ($<$className$>$\char`\"{}::\char`\"{})$^{\mbox{n}}$ $<$functionName$>$} \end{enumerate}
where n$>$0.

\begin{Desc}
\item[Note 1: ]Function arguments should be specified with correct types, i.e. 'fun(const std::string\&,bool)' or '()' to match any prototype. \end{Desc}
\begin{Desc}
\item[Note 2:]Member function modifiers (like 'const' and 'volatile') are required to identify the target, i.e. 'func(int) const' and 'fun(int)' target different member functions. \end{Desc}
\begin{Desc}
\item[Note 3: ]For JavaDoc compatibility a \# may be used instead of a :: in the patterns above. \end{Desc}
\begin{Desc}
\item[Note 4:]In the documentation of a class containing a member foo, a reference to a global variable is made using foo, whereas \#foo will link to the member.\end{Desc}
For non overloaded members the argument list may be omitted.

If a function is overloaded and no matching argument list is specified (i.e. pattern 2 or 6 is used), a link will be created to the documentation of one of the overloaded members.

For member functions the class scope (as used in patterns 4 to 7) may be omitted, if: \begin{enumerate}
\item The pattern points to a documented member that belongs to the same class as the documentation block that contains the pattern. \item The class that corresponds to the documentation blocks that contains the pattern has a base class that contains a documented member that matches the pattern. \end{enumerate}
\hypertarget{autolink_linkother}{}\section{Links to variables, typedefs, enum types, enum values and defines.}\label{autolink_linkother}
All of these entities can be linked to in the same way as described in the previous section. For sake of clarity it is advised to only use patterns 3 and 7 in this case.

\begin{Desc}
\item[Example:]

\begin{VerbInclude}\begin{verbatim}/*! \file autolink.cpp
  Testing automatic link generation.
  
  A link to a member of the Test class: Test::member, 
  
  More specific links to the each of the overloaded members:
  Test::member(int) and Test#member(int,int)

  A link to a protected member variable of Test: Test#var, 

  A link to the global enumeration type #GlobEnum.
 
  A link to the define #ABS(x).
  
  A link to the destructor of the Test class: Test::~Test, 
  
  A link to the typedef ::B.
 
  A link to the enumeration type Test::EType
  
  A link to some enumeration values Test::Val1 and ::GVal2
*/

/*!
  Since this documentation block belongs to the class Test no link to 
  Test is generated.

  Two ways to link to a constructor are: #Test and Test().

  Links to the destructor are: #~Test and ~Test().
  
  A link to a member in this class: member().

  More specific links to the each of the overloaded members: 
  member(int) and member(int,int). 
  
  A link to the variable #var.

  A link to the global typedef ::B.

  A link to the global enumeration type #GlobEnum.
  
  A link to the define ABS(x).
  
  A link to a variable \link #var using another text\endlink as a link.
  
  A link to the enumeration type #EType.

  A link to some enumeration values: \link Test::Val1 Val1 \endlink and ::GVal1.

  And last but not least a link to a file: autolink.cpp.
  
  \sa Inside a see also section any word is checked, so EType, 
      Val1, GVal1, ~Test and member will be replaced by links in HTML.
*/

class Test
{
  public:
    Test();               //!< constructor 
   ~Test();               //!< destructor 
    void member(int);     /**< A member function. Details. */
    void member(int,int); /**< An overloaded member function. Details */

    /** An enum type. More details */
    enum EType { 
      Val1,               /**< enum value 1 */ 
      Val2                /**< enum value 2 */ 
    };                

  protected:
    int var;              /**< A member variable */
};

/*! details. */
Test::Test() { }

/*! details. */
Test::~Test() { }

/*! A global variable. */
int globVar;

/*! A global enum. */
enum GlobEnum { 
                GVal1,    /*!< global enum value 1 */ 
                GVal2     /*!< global enum value 2 */ 
              };

/*!
 *  A macro definition.
 */ 
#define ABS(x) (((x)>0)?(x):-(x))

typedef Test B;

/*! \fn typedef Test B
 *  A type definition. 
 */
\end{verbatim}
\end{VerbInclude}
 \end{Desc}
\hypertarget{autolink_resolving}{}\section{typedefs.}\label{autolink_resolving}
Typedefs that involve classes, structs and unions, like 

\footnotesize\begin{verbatim}
typedef struct StructName TypeName
\end{verbatim}
\normalsize
 create an alias for StructName, so links will be generated to StructName, when either StructName itself or TypeName is encountered.

\begin{Desc}
\item[Example:]

\begin{VerbInclude}\begin{verbatim}/*! \file restypedef.cpp
 * An example of resolving typedefs.
 */

/*! \struct CoordStruct
 * A coordinate pair.
 */
struct CoordStruct
{
  /*! The x coordinate */
  float x;
  /*! The y coordinate */
  float y;
};

/*! Creates a type name for CoordStruct */ 
typedef CoordStruct Coord;

/*! 
 * This function returns the addition of \a c1 and \a c2, i.e:
 * (c1.x+c2.x,c1.y+c2.y)
 */
Coord add(Coord c1,Coord c2)
{
}
\end{verbatim}
\end{VerbInclude}
  \end{Desc}
